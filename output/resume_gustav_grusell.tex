% Copyright 2013 Christophe-Marie Duquesne <chmd@chmd.fr>
% Copyright 2014 Mark Szepieniec <http://github.com/mszep>
% 
% ConText style for making a resume with pandoc. Inspired by moderncv.
% 
% This CSS document is delivered to you under the CC BY-SA 3.0 License.
% https://creativecommons.org/licenses/by-sa/3.0/deed.en_US

\startmode[*mkii]
  \enableregime[utf-8]  
  \setupcolors[state=start]
\stopmode

\setupcolor[hex]
\definecolor[titlegrey][h=757575]
\definecolor[sectioncolor][h=397249]
\definecolor[rulecolor][h=BBBBBB]

% Enable hyperlinks
\setupinteraction[state=start, color=sectioncolor]

\setuppapersize [A4][A4]
\setuplayout    [width=middle, height=middle,
                 backspace=20mm, cutspace=0mm,
                 topspace=10mm, bottomspace=20mm,
                 header=0mm, footer=0mm]

%\setuppagenumbering[location={footer,center}]

\setupbodyfont[11pt, helvetica]

\setupwhitespace[medium]

\setupblackrules[width=31mm, height=1mm, color=rulecolor]

\setuphead[chapter]      [style=\tfd]
\setuphead[section]      [style=\tfd\bf, color=titlegrey, align=middle]
\setuphead[subsection]   [style=\tfb\bf, color=sectioncolor, align=right,
                          before={\leavevmode\blackrule\hspace}]
\setuphead[subsubsection][style=\bf]

\setuphead[chapter, section, subsection, subsubsection][number=no]

%\setupdescriptions[width=10mm]

\definedescription
  [description]
  [headstyle=bold, style=normal,
   location=hanging, width=18mm, distance=14mm, margin=0cm]

\setupitemize[autointro, packed]    % prevent orphan list intro
\setupitemize[indentnext=no]

\defineitemgroup[enumerate]
\setupenumerate[each][fit][itemalign=left,distance=.5em,style={\feature[+][default:tnum]}]

\setupfloat[figure][default={here,nonumber}]
\setupfloat[table][default={here,nonumber}]

\setuptables[textwidth=max, HL=none]
\setupxtable[frame=off,option={stretch,width}]

\setupthinrules[width=15em] % width of horizontal rules

\setupdelimitedtext
  [blockquote]
  [before={\setupalign[middle]},
   indentnext=no,
  ]


\starttext

\section[title={Gustav Grusell},reference={gustav-grusell}]

\thinrule

\startblockquote
Highly experienced developer with a keen interest in video encoding and
streaming technologies.
\stopblockquote

\thinrule

\subsection[title={Experience},reference={experience}]

\startdescription{2015 -}
  {\bf Developer and scrum master, Team Videocore, Sveriges Television
  AB}
\stopdescription

Team videocore at SVT is responsible for the transcoding and publishing
of all Video On Demand (VOD) content for SVT's online platforms. As a
developer, I take part in development and maintenance of a system of
microservices that handles transcoding, packaging and publishing of VOD
content.

Starting in 2019, I also have the role of scrum master in team
videocore. As a scrum master, I am responsible for helping the team
maintain and improve the agile process, and for ensuring the team can
perform at its highest level.

During my time at SVT I have acquired both solid knowledge about, as
well as a keen interest in, video encoding and streaming technologies. I
have worked a lot with java, kotlin and spring boot, and also with js
and react. My role as scrum master has given me good understanding of
agile processes and team dynamics.

Some of the highlights from my time at SVT:

\startitemize
\item
  Building
  \useURL[url1][https://github.com/svt/encore][][Encore]\from[url1], our
  own open source transcoding solution based on FFmpeg, and migrating to
  encore from our previous proprietary solution.
\item
  Migrating the video workflow from a proprietary service for
  dash/hls-packaging to handling packaging ourselves with shaka
  packager.
\item
  Implementing hevc-transcoding to increase visual quality and decrease
  bandwidth
\item
  Development of a machine-learning based tool for end credits
  detection.
\stopitemize

\startdescription{2014 - 2015}
  {\bf Senior Lead Developer, Bisnode Sverige AB}
\stopdescription

Development of a system for distribution of business data. I have taken
big responsibility in development of new subsystems. Development was
done using Java, Groovy and Spring. I worked as part of a small agile
team.

\startdescription{2010 - 2014}
  {\bf Java developer, Smartstream Technologies GmbH, Austria}
\stopdescription

Development of account reconciliations systems for customers in finance.
System was first deployed on Gigaspaces but was then migrated to JBoss
Fuse. I lead the work in optimizing the software for resource
utilization and throughput, and the migration from Gigaspaces XAP to
Fuse Fabric. Development was done with Java and Spring.

\startdescription{2006 - 2008}
  {\bf Software Consultant, Gnistra AB}
\stopdescription

Worked with Java development with JSF, hibernate, as well as development
of industrial ultrasonic measuring systems with LabView. Independent
work in smaller projects in close collaboration with customers.

\startdescription{2004 - 2006}
  {\bf Research scientist, environmental modeling, ESS GmbH, Austria}
\stopdescription

ESS is a software company specialized in systems for environmental data
and environmental modeling. The work was mainly development of software
processing for simulating environmental processed, and development of
web-based user interface for the simulations. The work included
numerical modeling in c++, user interface development in PHP and JAVA,
and database management with MySQL.

\subsection[title={Education},reference={education}]

\startdescription{2013 - 2014}
  {\bf Master's program in Social-Ecological Resilience for Sustainable
  Development, Stockholm Resilience Center}
\stopdescription

\startdescription{2009}
  {\bf German language, Vienna university}
\stopdescription

\startdescription{1996 - 2003}
  {\bf Master's Program in Environmental and Water Engineering, Uppsala
  University}
\stopdescription

\subsection[title={Technical
Experience},reference={technical-experience}]

\startdescription{{\bf Programming Languages}}
  Java, Kotlin, JavaScript, c++, Groovy, python, Labview/G, bash,
  emacs-lisp, clojure, R
\stopdescription

\startdescription{{\bf Frameworks}}
  Spring-boot, React
\stopdescription

\startdescription{{\bf Technologies}}
  Docker, Kubernetes, FFmpeg, Apache Kafka, MySql/MariaDB, Redis,
  Elastic Search, Cassandra, Oracle, MSSQL, Apache cassandra
\stopdescription

\subsection[title={Open Source
Projects},reference={open-source-projects}]

\startdescription{{\bf Encore}}
  A scalable video transcoding tool, built around FFmpeg. Built by the
  videocore team at SVT.
  \useURL[url2][https://github.com/svt/encore]\from[url2]
\stopdescription

\startdescription{{\bf Vivict}}
  An easy to use in-browser tool for subjective comparison of the visual
  quality of different encodings of the same video source. Built with
  react and js. I created this because I saw a need for simple free tool
  for comparing video quality.
  \useURL[url3][https://github.com/svt/vivict]\from[url3]
\stopdescription

\startdescription{{\bf Vivict++}}
  An easy to use desktop tool for subjective comparison of the visual
  quality of different encodings of the same video source. Based on
  ffmpeg/libav, it supports a much wider range of codecs and formats
  than vivict. Built in c++.
  \useURL[url4][https://github.com/svt/vivictpp]\from[url4]
\stopdescription

\subsection[title={Human Languages},reference={human-languages}]

\startitemize[packed]
\item
  Swedish - native speaker
\item
  English - fluent
\item
  German - fluent
\stopitemize

\thinrule

\startblockquote
\useURL[url5][mailto:gustav.grusell@gmail.com][][gustav.grusell@gmail.com]\from[url5]
• +46 736401141 www.linkedin.com/in/gustavgrusell •
https://github.com/grusell/
\stopblockquote

\stoptext
